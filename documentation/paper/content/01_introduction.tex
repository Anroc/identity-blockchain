\chapter{Introduction}
\label{cha:introduction}

Our world is changing. We are shifting more and more to digitization every aspect of our daily life.
While some transitions were quite easily done, like the digitization of news, photos or social life,
there are some categories that are restricted by federal laws. For example everything that is related with processing,
storing or evaluating personal data. There are restrictions on confidentiality, integrity and of cause also privacy.
A German law for example states that data needs to be protected from any kind of tampering while its transmission.
To meet this requirements some could think about hashing and encrypting message authentication codes to ensure its
integrity and authenticity while being transmitted on the wire.

However, with the raise of Bitcoin and the recently media attention, came a big hype for blockchain based technologies,
which provide per default a strong tamper proof property.

Personal data is distributed and untransparent for the user to view.\cite[p. 1]{uPortWhitePaper}

Blockchain and decentralization help to push the ownership back to the individuals so that they are in control of their
data. This principle is often referred as \textit{self-sovereignity}.\cite[p. 1]{uPortWhitePaper}

% User-centric: Data is organized and managed by user but stored at third parties.
%  2008, Kim Cameron, Reinhard Posch and Kai Rannenberg describes “A User-Centric Identity Metasystem”.
It details an abstracted design for a
% cited in https://sovrin.org/wp-content/uploads/2017/07/The-Inevitable-Rise-of-Self-Sovereign-Identity.pdf page 9

Self Sovereign individual control, security, and full portability
% https://sovrin.org/wp-content/uploads/2017/07/The-Inevitable-Rise-of-Self-Sovereign-Identity.pdf page 9

\section{Motivation}
