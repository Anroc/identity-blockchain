\chapter{Introduction}
\label{cha:introduction}
\section{Digital identity and the current Problem}

The todays problem with a digital identity lies first of all in division of the data which makes a identity. With several data providers the data is heterogeneous and fragmented within multiple servers and each requires its own password and user name. So the best solution for an exchange of identity informations might be, to let each provider store the data of their users on one centralized server, controlled by one trusted party. Not only is this a issue regarding security, because one system can be more easily corrupted but also all user has to trust the one provider.
Nevertheless these issues aren't simply solved, a overall system or framework to get identity information from several providers and make their exchange more easy is desirable. 

\section{Digital Identity}
Before a solution for these issues can be found a definition for the term identity is needed to identifiy what data can be collected. \cite{digitalIdentityDefinition} describes it as follows: A digital identity corresponds to the electronic information associated with an individual
in a particular identity system. Identity systems are used by online service providers to authenticate and authorize users [...]
With an increasing amount of online services and application a user creates more and more identites within different systems. Therefore the importance of managing those identites is growing \cite{managingIdentity}. 
It enables the cooperation between multiple data providers to enable and improve their services and better target their products and services. Problem arises when individuals no longer have control over how their information is collected and used, which not seldom containing sensitive attributes. To solve this problem high requirements on privacy and confidentiality are prescribed by law.

\section{GDPR}

General Data Protection Regulation (GDPR) is a regulation that provides rules how to process and collect personal data by private companies and authorities across the EU, to strengthen data protection. It's furthermore secures free data traffic across all members within the EU. \cite{gdpr}

-> “data protection by design and default
-> pseudonymization (decoupling data from individual identity) and data minimization (sharing only absolutely necessary data points)
-> the GDPR aims to create an environment maximally friendly to individuals needing to transfer data between different vendors, governments, and institutions at their own discretion.

-> principles of digital self-sovereignty articulated by Christopher Allen in an influential 2016 essay.


security -> decentralized technologies, key cryptography
blockchain (private, public, permission ledger)
how it works
peer to peer
ethereum

trusted party
GDPR
self suvireign

-> provider halten gerne daten und sammeln gerne daten

-> peer to pper
-> Bitcoin has demonstrated
in the financial space that trusted, auditable computing is possible
using a decentralized network of peers accompanied by a public
ledger

->While we all reap the benefits of a data-driven society, there
is a growing public concern about user privacy. Centralized
organizations – both public and private, amass large quantities
of personal and sensitive information. Individuals have little or
no control over the data that is stored about them and how it
is used. In recent years, public media has repeatedly covered
controversial incidents related to privacy. Among the better
known examples is the story about government surveillance
[2], and Facebook’s large-scale scientific experiment that was
apparently conducted without explicitly informing participants



Personal data is distributed and untransparent for the user to view.\cite[p. 1]{uPortWhitePaper}

Blockchain and decrentralization help to push the ownership back to the indiviuals so that they are in controll of their data. This principle is often refered as \textit{self-sovereignity}.\cite[p. 1]{uPortWhitePaper}





% User-centric: Data is organized and managed by user but stored at thrid parties. 
%  2008, Kim Cameron, Reinhard Posch and Kai Rannenberg describes “A User-Centric Identity Metasystem”. It details an abstracted design for a
% cited in https://sovrin.org/wp-content/uploads/2017/07/The-Inevitable-Rise-of-Self-Sovereign-Identity.pdf page 9

Self Sovereign individual control, security, and full portability
% https://sovrin.org/wp-content/uploads/2017/07/The-Inevitable-Rise-of-Self-Sovereign-Identity.pdf page 9
