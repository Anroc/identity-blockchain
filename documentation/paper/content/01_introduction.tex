\chapter{Introduction}
\label{cha:introduction}
\section{Digital identity and the current Problem}
A digital identity corresponds to the electronic information associated with an individual
in a particular identity system. Identity systems are used by online service providers to authenticate and authorize users [...].\cite{digitalIdentity}

The todays problem with a digital identity lies first of all in division of the data which makes a identity. With several data providers the data is heterogeneous and fragmented within multiple servers and each requires its own password and user password. So the best solution might be, to let each provider store its data in a centrilized, controlled by the government 

-> einheitliche Plattform für übersicht
-> einheitlich Plattform für austausch von informationen auch für service provider



Personal data is distributed and untransparent for the user to view.\cite[p. 1]{uPortWhitePaper}

Blockchain and decrentralization help to push the ownership back to the indiviuals so that they are in controll of their data. This principle is often refered as \textit{self-sovereignity}.\cite[p. 1]{uPortWhitePaper}





% User-centric: Data is organized and managed by user but stored at thrid parties. 
%  2008, Kim Cameron, Reinhard Posch and Kai Rannenberg describes “A User-Centric Identity Metasystem”. It details an abstracted design for a
% cited in https://sovrin.org/wp-content/uploads/2017/07/The-Inevitable-Rise-of-Self-Sovereign-Identity.pdf page 9

Self Sovereign individual control, security, and full portability
% https://sovrin.org/wp-content/uploads/2017/07/The-Inevitable-Rise-of-Self-Sovereign-Identity.pdf page 9
