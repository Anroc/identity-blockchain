\chapter*{Abstract}
\label{cha:abstract}

In this thesis, we show an innovative approach to manage your own digital identity in the context
of the newly released GDPR using blockchain for transparency of ownership of data.
Compared with other blockchain-based solutions uPort, MedRec and Hawk,
this approach is GDPR compliant as no personal information is stored in the blockchain, solves bottlenecks and critical
trains of thought.
It is self-sovereign by storing all identity oinformation of users and routing each request for population, evaluation and
changing of claims to the user.
In addition it provides a thorough look into the implementation of a prototype, its reasonings, examines challanges
how to establish and maintain trust in the system and how to maintain the blockchain's transparency properties in a
senseful way without exposing any identity information.

%Micha: mein aufschlag für einen abstract
With upcoming strict digital identity regulation frameworks such as the GDPR\cite{gdpr} it is necessary to 
rethink current approaches on how digital identites are being handled. The continuous rise of blockchain technologies and the concepts that power them offer solutions to some of the upcoming problems. Necessitating a groundbreaking shift in established paradigms and structures.
Instead of just securely handling user data it is now imperative to also offer mechanisms to enable users to act in a self-sovereign fashion. Furthermore companies are to surrender control over each individuals data back to the owning entity. 
We propose a newly developed system, \projectName{}, capable of handling user data while still being GDPR compliant. \projectName{} is based upon previous and still continuing work such as uPort\cite{uPortWhitePaper}, Soverin\cite{} and in large parts medRec\cite{azaria2016medrec}.
\projectName{} governs the blockchain in such a fashion that only permissioned entities are allowed to write to it. These consist of a trusted central authority in the form of the government and the actual citizens that are part of the system. 
In a nutshell \projectName{} aims to utilize the blockchain as a public ledger containing transactions without containing a users specific claims or attributes. Instead each user holds his own private data and acts fully sovereign by granting or denying other parties access to it.
In the following chapters we examine the challenges associated with utilizing blockchains for this purpose, how to establish trust in a system with pseudonominized entities and how to use the blockchains transparency to suit the need for verification, while adhering to the principles outlined within the GDPR\cite{gdpr}.
Lastly we will explain our methodology, concept and implementation. Further we will expand upon our reasoning behind the choices we have taken and critically reevaluate our results and approaches. 