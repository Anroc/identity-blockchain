\chapter*{Abstract}
\label{cha:abstract}

%Micha: mein aufschlag für einen abstract
With upcoming strict digital identity regulation frameworks such as the GDPR\cite{gdpr} it is necessary to 
rethink current approaches on how digital identities are handled.
The continuous rise of blockchain technologies and the concepts that power
them offer solutions to some of the upcoming problems.
Necessitating a groundbreaking shift in established paradigms and structures.
Instead of just securely handling user data it is now imperative to also offer mechanisms to enable users to act
in a self-sovereign fashion.
Furthermore companies are to surrender control over each individuals data back to the owning entity.
We propose a newly developed system, \projectName{}, capable of handling user data while still being GDPR compliant.
\projectName{} is based upon previous and still continuing work such as uPort\cite{uPortWhitePaper},
Soverin\cite{}, Hawk\cite{} and in large parts medRec\cite{azaria2016medrec}.
\projectName{} governs the blockchain in such a fashion that only permissioned entities are allowed to write to it.
These consist of a trusted central authority in the form of the government and the actual citizens
that are part of the system.
In a nutshell \projectName{} aims to utilize the blockchain as a public ledger containing transactions without
containing a users specific claims or attributes.
Instead each user holds his own private data and acts fully self-sovereign by granting or denying other parties access
to it while keeping full control over the data.
In the following chapters we examine the challenges associated with utilizing blockchains for this purpose,
how to establish trust in a system with pseudonymized entities and how to use the blockchains transparency to suit the
need for verification, while adhering to the principles outlined within the GDPR\cite{gdpr}.
Lastly we will explain our methodology, concept and implementation.
Further we will expand upon our reasoning behind the choices we have taken and critically reevaluate our results and approaches.